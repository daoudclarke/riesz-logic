% Bismillahi-r-Rahmani-r-Rahim
% In the Name of God the Merciful, the Compassionate

\documentclass[preprint,leqno]{elsarticle}
%\documentclass{article}

\author{Daoud Clarke}
\date{\today}
\title{The Logic of Vector Lattices}

%\usepackage{times}
%\usepackage{mathptmx}
%\usepackage{newtxtext,newtxmath}

\usepackage{amsmath}

\begin{document}

\maketitle


\section{Introduction}

We introduce Riesz Logic, a variant of fuzzy logic whose models are
lattice ordered groups, of which the most familiar examples are vector
lattices, or Riesz spaces.
% Our logic has axioms very similar to those
% of fuzzy logics, in particular Basic Fuzzy Logic (BL), although the
% models are quite different.
%
% % Axioms:
% % These two together allow proof of first five BAL axioms
% P((x => y) => ((y => z) => (x => z))) # label(BL1).
% P(((y => z) => (x => z)) => (x => y)) # label(BL1a).
%
% % Needed to prove BALPI and BALPa
% P(x ^ y => x)                         # label(BL2).
%
% % Needed to prove BALO
% P(x ^ y => y ^ x)                     # label(BL3).
%
% % Monotonicity of conjunction
% P((x => y) => (z ^ x => z ^ y)).
%
Our logic has a single derivation rule, modus ponens
\begin{equation}\tag{MP}
\frac{\phi, \phi \rightarrow \psi}{\psi},
\end{equation}
the following axioms from Basic Fuzzy Logic:
\begin{align}
  \tag{BL1} &(\phi \rightarrow \psi) \rightarrow ((\psi \rightarrow \chi)
  \rightarrow (\phi \rightarrow \chi))\\
  \tag{BL2} &\phi \land \psi \rightarrow \phi\\
  \tag{BL3} &\phi \land \psi \rightarrow \psi \land \phi,
\end{align}
and the following new axioms:
\begin{align}
  \tag{R1}  &((\psi \rightarrow \chi) \rightarrow (\phi \rightarrow
  \chi)) \rightarrow (\phi \rightarrow \psi)\\
  \tag{R2}  &(\phi \rightarrow \psi) \rightarrow (\chi \land \phi
  \rightarrow \chi \land \psi).
\end{align}
In this paper we prove that the models for this logic

%  In particular there is no ``True'' or
% ``False'' constant, instead there is a zero value which indicates
% maximal uncertainty about a proposition. We demonstrate that our logic
% is equivalent to the logic BAL of

The motivation for our work is recent work in natural language
processing, in which the meaning of words is determined by the
contexts in which they occur. This notion of ``distributional
semantics'' has its origin in the work of Firth \cite{Firth:57} and
Harris \cite{Harris:68}.

\bibliographystyle{elsarticle-harv}
\bibliography{JW2012}


\section{Motivation}



\end{document}