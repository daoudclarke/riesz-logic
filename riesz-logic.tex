% Bismillahi-r-Rahmani-r-Rahim
% In the Name of God the Merciful, the Compassionate

\documentclass[preprint,leqno]{elsarticle}
%\documentclass{article}

\author{Daoud Clarke}
\date{\today}
\title{The Logic of Vector Lattices}

%\usepackage{times}
%\usepackage{mathptmx}
%\usepackage{newtxtext,newtxmath}

\usepackage{amsmath}
\usepackage{stmaryrd}
\usepackage{cmll}
%\usepackage{mathabx}

\newdefinition{definition}{Definition}
\newproof{proof}{Proof}

%\newcommand{\interp}[1]{\ldbrack #1 \rdbrack}
\newcommand{\interp}[1]{\llbracket #1 \rrbracket}

\begin{document}

\maketitle


\section{Introduction}

We introduce Riesz Logic, a variant of fuzzy logic whose models are
abelian lattice ordered groups, of which the most familiar examples
are vector lattices, or Riesz spaces.
% Our logic has axioms very similar to those
% of fuzzy logics, in particular Basic Fuzzy Logic (BL), although the
% models are quite different.
%

% % These two together allow proof of first five BAL axioms
% P((x => y) => ((y => z) => (x => z)))         # label(R1).
% P(((y => z) => (x => z)) => (x => y))         # label(R2).

% % Needed to prove BALPI and BALPa
% P(x => x v y)                                 # label(R3).

% % Needed to prove BALO
% P(x v y => y v x)                             # label(R4).

% %% % Adapted from BALP
% P((x v y) v y => x v y)                       # label(R5).

% % Definition of zero
% P(0 => (z => z))                              # label(R6a).
% P((z => z) => 0)                              # label(R6b).

% %% % Adapted from BALO
% P(((x => y) v 0 => (y => x) v 0) => (y => x)) # label(R7a).
% P((y => x) => ((x => y) v 0 => (y => x) v 0)) # label(R7b).

%
Our logic has a single derivation rule, modus ponens
\begin{equation}\tag{MP}
\frac{\phi, \phi \rightarrow \psi}{\psi},
\end{equation}
and the following axioms:
\begin{align}
  \tag{R1a} &(\phi \rightarrow \psi) \rightarrow ((\psi \rightarrow \chi)
  \rightarrow (\phi \rightarrow \chi))\\
  \tag{R1a} &((\psi \rightarrow \chi) \rightarrow (\phi \rightarrow
  \chi)) \rightarrow (\phi \rightarrow \psi)\\
  \tag{R2} &\phi \rightarrow \phi \lor \psi\\
  \tag{R3} &\phi \lor \psi \rightarrow \psi \lor \phi,\\
  \tag{R4} &(\phi \lor \psi)\lor \psi \rightarrow \phi \lor \psi\\
  \tag{R5a} &0 \rightarrow (\phi \rightarrow \phi)\\
  \tag{R5b} &(\phi \rightarrow\phi) \rightarrow 0\\
  \tag{R6a} &((\phi \rightarrow \psi)\lor 0 \rightarrow (\psi \rightarrow \phi) \lor 0) \rightarrow (\psi \rightarrow \phi)\\
  \tag{R6b} & (\psi \rightarrow \phi) \rightarrow ((\phi \rightarrow \psi)\lor 0 \rightarrow (\psi \rightarrow \phi) \lor 0)
\end{align}
In this paper we prove the soundness and completeness of this logic
with respect to abelian lattice ordered groups, with formulas
interpreted as asserting positivity. In doing this, we relate Riesz
Logic to the Logic of Equilibrium \cite{Galli:04}.

%  In particular there is no ``True'' or
% ``False'' constant, instead there is a zero value which indicates
% maximal uncertainty about a proposition. We demonstrate that our logic
% is equivalent to the logic BAL of

\section{Motivation}

The motivation for our work is recent work in natural language
processing, in which the meaning of words is determined by the
contexts in which they occur. This notion of ``distributional
semantics'' has its origin in the work of Firth \cite{Firth:57} and
Harris \cite{Harris:68}, and the philosophy of Wittgenstein
\cite{Wittgenstein:53}.

\section{Interpretations}

We prove the soundness and completeness of Riesz logic with respect to
abelian lattice ordered groups.

\begin{definition}[Lattice Ordered Group]
  A partially ordered group is a tuple $\langle G, +, \le\rangle$ such
  that $\langle G, +\rangle$ is a group, and $\le$ is a partial order
  on $G$ such that if $u \le v$ then $u + w \le v + w$ and $w + u \le
  w + v$. If $\le$ is a lattice order, then $G$ is called a lattice
  ordered group. Where there is no confusion, we refer to the lattice
  ordered group $\langle G, +, \le\rangle$ as simply $G$. We denote
  the lattice meet and join by $\land$ and $\lor$ respectively.
\end{definition}

Riesz spaces are abelian lattice ordered groups where the group
operation is vector space addition, and the vector space zero is the
unit of the group.

An interpretation $\langle G, F\rangle$ for Riesz Logic
is an abelian lattice ordered group $G$ and a function $F$ that maps
variables in Riesz Logic to elements of $G$. A Riesz Logic formula $x$
has the interpretation $\interp{x}$ defined recursively as follows:
\begin{itemize}
\item $\interp{\phi} = F(\phi)$
\item $\interp{x \rightarrow y} = \interp{y} - \interp{x}$
\item $\interp{x \with y} = \interp{x} \land \interp{y}$
\end{itemize}
The formula $x$ is interpreted as asserting that $0 \le
\interp{x}$. Thus, for example, the formula $\phi \rightarrow \psi$ is
interpreted as the assertion $0 \le F(\psi) - F(\phi)$, or $F(\phi)
\le F(\psi)$. A formula $x$ is satisfiable if there is some
interpretation such that $0 \le \interp{x}$; it is a theorem or
tautology if $0 \le \interp{x}$ for all interpretations.

\section{Soundness}

Proving soundness of the logic amounts to proving the validity of the
rule and axioms.

%\subsection{Modus Ponens}

\begin{proof}[Modus Ponens]
If $0 \le F(\phi)$ and $0 \le F(\psi) - F(\phi)$ then
$0 \le F(\psi)$ by transitivity of $\le$.
\end{proof}

\begin{proof}[BL1 and R1]
  Since R1 is the converse of BL1, they may be taken together as
  asserting equality, by the antisymmetry of $\le$. Thus we need to
  show:
  \begin{align*}
    \interp{\phi \rightarrow \psi} & = \interp{(\psi \rightarrow
      \chi) \rightarrow (\phi \rightarrow \chi)}\\
    F(\psi) - F(\phi) & = \interp{\phi \rightarrow \chi} -
    \interp{\psi \rightarrow \chi}\\
    & = F(\chi) - F(\phi) - F(\chi) + F(\psi)\\
    & = F(\psi) - F(\phi).
  \end{align*}
\end{proof}
BL2 and BL3 are trivially seen to be properties of the partial
ordering.

\begin{proof}[R2]
\begin{align*}
  \interp{\phi \with \chi \rightarrow \psi} &\le \interp{(\phi \rightarrow \psi) \with (\chi \rightarrow \psi)}\\
  F(\psi) - F(\phi)\land F(\chi) &\le (F(\psi) - F(\phi)) \land
  (F(\psi) - F(\chi))
\end{align*}
\end{proof}

\section*{References}

\bibliographystyle{elsarticle-harv}
\bibliography{JW2012}


\section{Motivation}



\end{document}