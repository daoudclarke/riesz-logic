% Bismillahi-r-Rahmani-r-Rahim
% In the Name of God the Merciful, the Compassionate

\documentclass[preprint,3p]{elsarticle}
%\documentclass{article}

\author{Daoud Clarke}
\date{\today}
\title{The Logic of Vector Lattices}

\usepackage{times}

\begin{document}

\maketitle


\section{Introduction}

In this paper we introduce a logic whose models are lattice
ordered groups, of which the most familiar examples are the vector
lattices, or Riesz spaces. Our logic has axioms very similar to those
of fuzzy logics, in particular Basic Fuzzy Logic (BL), although the
models are quite different.

Our logic has a single derivation rule, modus ponens

and the
following axioms:
% \begin{eqnarray}

% \end{eqnarray}

%  In particular there is no ``True'' or
% ``False'' constant, instead there is a zero value which indicates
% maximal uncertainty about a proposition. We demonstrate that our logic
% is equivalent to the logic BAL of

The motivation for our work is recent work in natural language
processing, in which the meaning of words is determined by the
contexts in which they occur. This notion of ``distributional
semantics'' has its origin in the work of Firth and Harris

\section{Motivation}



\end{document}